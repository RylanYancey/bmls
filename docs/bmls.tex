
\documentclass{report}

\usepackage{amsmath}
\usepackage[backend=biber, style=alphabetic,]{biblatex}
\usepackage[colorlinks=true, urlcolor=blue, linkcolor=black]{hyperref}

\addbibresource{docs.bib}

\title{ML Operators}
\author{BMLS Team}

\begin{document}

    \maketitle

    \tableofcontents

    \chapter{Introduction to BMLS}

    Basic Machine Learning Subsystems aims to provide highly performant implementations of common operations in machine learning \textit{and}
    their gradients. As the complexity and quantity of operators increases, BMLS provides a way to manage that complexity. We do this by 
    formally defining the \textit{forward} and \textit{backward} pass of each operator, peer reviewing those definitions, and providing extensive
    unit testing. This way, implementers can be absolutely certain the operation is correct.

        BMLS is \textit{not} a machine-learning library. It is a \textit{math} library on which machine learning libraries, such as for reverse-mode
        automatic differentiation, can be implemented.

        \paragraph{Data Formats}
        All Tensors in BMLS are assumed to be row-major, NCHW tensors where N = batch size, C = channels, H = height, and W = width. Rank 2 Tensors
        are assumed to be NC, Rank 3 are assumed to be CHW, and Rank 4 are assumed to be NCHW. 

        \paragraph{Design Patterns}
        All operations provided by BMLS have a forward operation \verb|name()| and a backwards operation \verb|name_wrt_xn()| for each 
        differentiable input. Operation inputs are \verb|x1|, \verb|x2|, \verb|x3|, etc, followed by an output \verb|y| and gradients are denoted 
        \verb|g1|, \verb|g2|, etc. Inputs are always \verb|*const| and outputs are always \verb|*mut|.
    
        \paragraph{Features}
        BMLS provides element wise operators, activation functions, higher rank operators, loss functions, and gradients for all. 
        It does not, however, implement any error handling. You are expected to implement error handling on an as-needed basis. 

    \section{Contributors}
    \begin{itemize}
        \item Rylan W. Yancey
    \end{itemize}

    \section{Peer Reviewers}
    \begin{itemize}
        \item J. Leon Ballentine
    \end{itemize}

    \chapter{Element-Wise Operators}

        \section{Add}
            The Addition Operator is defined as $f(u,v) = u + v$. To find the gradient w.r.t u and v, 
            we will use the sum rule, which is defined as: 

            $$\frac{\delta}{\delta{x}}(u + v) = \frac{\delta{u}}{\delta{x}} + \frac{\delta{v}}{\delta{x}}$$

            To find the gradient w.r.t u, we will set u as our x and treat v as a constant, 
            which gives us the following:

            $$\frac{\delta}{\delta{u}}(u + v) = \frac{\delta{u}}{\delta{u}} + \frac{\delta{v}}{\delta{u}} = 1$$

            To find the gradient w.r.t v, we will set v as our x and treat u as a constant, 
            which gives us the following:

            $$\frac{\delta}{\delta{v}}(u + v) = \frac{\delta{u}}{\delta{v}} + \frac{\delta{v}}{\delta{v}} = 1$$

            Therefore, we can say that the gradient w.r.t u is 1, and the gradient w.r.t v is 1.

        \section{Sub}
            The Subtraction Operator is defined as $f(u,v) = u - v$. To find the gradient w.r.t u and v, 
            we will use the subtraction rule, which is defined as: 
        
            $$\frac{\delta}{\delta{x}}(u - v) = \frac{\delta{u}}{\delta{x}} - \frac{\delta{v}}{\delta{x}}$$
        
            To find the gradient w.r.t u, we will set u as our x and treat v as a constant, 
            which gives us the following:
        
            $$\frac{\delta}{\delta{u}}(u - v) = \frac{\delta{u}}{\delta{u}} - \frac{\delta{v}}{\delta{u}} = 1$$
        
            To find the gradient w.r.t v, we will set v as our x and treat u as a constant, 
            which gives us the following:
        
            $$\frac{\delta}{\delta{v}}(u - v) = \frac{\delta{u}}{\delta{v}} - \frac{\delta{v}}{\delta{v}} = -1$$
        
            Therefore, we can say that the gradient w.r.t u is 1, and the gradient w.r.t v is -1. 

        \section{Mul}
            The Multiplication Operator is defined as $f(u, v) = uv$. To find the gradient w.r.t 
            u and v, we will use the product rule of derivatives, which is defined as:

            $$\frac{\delta}{\delta{x}}(uv) = v\frac{\delta{u}}{\delta{x}} + u\frac{\delta{v}}{\delta{x}}$$

            To find the gradient w.r.t u, we will set u as our x and treat v as a constant, 
            which gives us the following:

            $$\frac{\delta}{\delta{u}}(uv) = v\frac{\delta{u}}{\delta{u}} + u\frac{\delta{v}}{\delta{u}} = v$$

            To find the gradient w.r.t v, simply do the same, this time setting v as x and treat u as constant.

            $$\frac{\delta}{\delta{v}}(uv) = v\frac{\delta{u}}{\delta{v}} + u\frac{\delta{v}}{\delta{v}} = u$$

            Therefore, we can say that the gradient w.r.t u is v, and the gradient w.r.t v is u. 

        \section{Div}
            The Division Operator is defined as $f(u, v) = \frac{u}{v}$. To find the gradient w.r.t x and y, we will
            use the quotient rule of derivatives, which is defined as:
    
            $$\frac{\delta}{\delta{x}}(\frac{u}{v}) = \frac{(u\frac{\delta}{\delta{x}})v - u(v\frac{\delta}{\delta{x}})}{v^2}$$
    
            To find the gradient w.r.t u, we will set u as our x and treat v as constant, which gives us the following:
    
            $$\frac{\delta}{\delta{u}}(\frac{u}{v}) = \frac{v\frac{\delta{u}}{\delta{u}} - u\frac{\delta{v}}{\delta{u}}}{v^2} 
            = \frac{v - u\frac{0}{\delta{u}}}{v^2} = \frac{v}{v^2} = \frac{1}{v}$$
    
            To find the gradient w.r.t v, we will set v as our x and treat u as constant, which gives us the following:
    
            $$\frac{\delta}{\delta{v}}(\frac{u}{v}) = \frac{v\frac{\delta{u}}{\delta{v}} - u\frac{\delta{v}}{\delta{v}}}{v^2} 
            = \frac{v\frac{0}{\delta{v}} - u}{v^2} = -\frac{u}{v^2}$$
    
            Therefore, we can say that the gradient w.r.t u is $\frac{1}{v}$, and the gradient w.r.t v is $-\frac{u}{v^2}$.

        \section{Axis Add}
            The Axis Add Operator iterates over an axis of input $A$ and adds the $B$ value corresponding that that index of the axis.
            $B$ is a vector that has the same length as the axis to be added to, and A is a tensor.
            In terms of $B_i$ over axis $A_i$, where $B_i$ is a value and $A_i$ is a vector of length $n$. 
            $$f(A_i, B_i) = \sum_{j=0}^{n} C{ij} = A_{ij} + B_i$$
            
            \paragraph{Gradient}
            By definition, the axis add operator is a kind of element-wise addition. To find the gradient w.r.t. $B_i$, we can apply the sum rule.
            $$\frac{\delta}{\delta{B_i}}(\sum_{j=0}^{n} A_{ij} += B_i) = \sum_{i=0}^{n} 1 = n$$
            Therefore, we can say the gradient w.r.t. $B_i$ is the sum the gradients of all its additions. 
            The gradient w.r.t. $A$ is 1, since it is a simple element-wise addition. 

            We must apply the chain rule to this operation to backpropogate with it. To do so, 
            we will multiply the gradient w.r.t. output C $\frac{\delta{L}}{\delta{C_i}}$. 
            $$\frac{\delta{L}}{\delta{B_i}} = \sum_{i=0}^{n} \frac{\delta{L}}{\delta{C_i}}$$

        \section{Axis Sub}
            The Axis Add Operator iterates over an axis of input $A$ and subtracts the $B$ value corresponding that that index of the axis.
            $B$ is a vector that has the same length as the axis to be added to, and A is a tensor.
            In terms of $B_i$ over axis $A_i$, where $B_i$ is a value and $A_i$ is a vector of length $n$. 
            $$f(A_i, B_i) = \sum_{j=0}^{n} C_{ij} = A_{ij} - B_i$$

            \paragraph{Gradient}
            By definition, the axis sub operator is a kind of element-wise subtraction. To find the gradient w.r.t. $B_i$, 
            we can apply the subtraction rule.
            $$\frac{\delta}{\delta{B_i}}(\sum_{j=0}^{n} C_{ij} = A_{ij} - B_i) = \sum_{i=0}^{n} -1 = n$$
            Therefore, we can say the gradient w.r.t. $B_i$ is the sum the gradients of all its subtractions. 
            The gradient w.r.t. $A$ is 1, since it is a simple element-wise subtraction.

            We must apply the chain rule to this operation to backpropogate with it. To do so, 
            we will multiply the gradient w.r.t. output C $\frac{\delta{L}}{\delta{C_i}}$. 
            $$\frac{\delta{L}}{\delta{B_i}} = \sum_{i=0}^{n} -\frac{\delta{L}}{\delta{C_i}}$$

        \section{Axis Mul}
            The Axis Mul Operator iterates over an axis of input $A$ and multiplies the $B$ value corresponding that that index of the axis.
            $B$ is a vector that has the same length as the axis to be added to, and A is a tensor.
            In terms of $B_i$ over axis $A_i$, where $B_i$ is a value and $A_i$ is a vector of length $n$. 
            $$f(A_i, B_i) = \sum_{j=0}^{n} C_{ij} = A_{ij}B_i$$

            \paragraph{Gradient w.r.t. B}
            By definition, the axis mul operator is a kind of element-wise multiplication. To find the gradient w.r.t. $B_i$, 
            we can apply the product rule.
            $$\frac{\delta}{\delta{B_i}}(\sum_{j=0}^{n} C_{ij} = A_{ij}B_i) = \sum_{i=0}^{n} A_{ij}$$
            Therefore, we can say the gradient w.r.t. $B_i$ is the sum the gradients of all its multiplications.additions.

            We must apply the chain rule to this operation to backpropogate with it. To do so, 
            we will multiply the gradient w.r.t. output C $\frac{\delta{L}}{\delta{C_i}}$. 
            $$\frac{\delta{L}}{\delta{B_i}} = \sum_{i=0}^{n} \frac{\delta{L}}{\delta{C_i}} A_{ij}$$

            \paragraph{Gradient w.r.t. A}
            Applying the same logic, we can conclude that the gradient w.r.t. $A_{ij}$ is the value in $B$ it was multiplied by.
            $$\frac{\delta}{\delta{A_{ij}}} = (A_{ij}B_i) = B_i$$

        \section{Axis Div}
            The Axis Div Operator iterates over an axis of input $A$ and multiplies the $B$ value corresponding that that index of the axis.
            $B$ is a vector that has the same length as the axis to be added to, and A is a tensor.
            In terms of $B_i$ over axis $A_i$, where $B_i$ is a value and $A_i$ is a vector of length $n$. 
            $$f(A_i, B_i) = \sum_{j=0}^{n} C_{ij} = \frac{A_{ij}}{B_i}$$

            \paragraph{Gradient w.r.t. B}
            By definition, the axis div operator is a kind of element-wise division. To find the gradient w.r.t. $B_i$, 
            we can apply the division rule.
            $$\frac{\delta}{\delta{B_i}}(\sum_{j=0}^{n} C_{ij} = \frac{A_{ij}}{B_i}) = \sum_{i=0}^{n} \frac{A_{ij}}{B_i^2}$$
            Therefore, we can say the gradient w.r.t. $B_i$ is the sum the gradients of all its divisions. 

            We must apply the chain rule to this operation to backpropogate with it. To do so, 
            we will multiply the gradient w.r.t. output $C_i$ $\frac{\delta{L}}{\delta{C_i}}$. 
            $$\frac{\delta{L}}{\delta{B_i}} = \sum_{i=0}^{n} \frac{\delta{L}}{\delta{C_i}} \frac{A_{ij}}{B_i^2}$$

            \paragraph{Gradient w.r.t. A}
            Applying the same logic, we can conclude that the gradient w.r.t. $A_{ij}$ is the value in $B$ it was multiplied by.
            $$\frac{\delta}{\delta{A_{ij}}}(\frac{A_{ij}}{B_i}) = \frac{1}{B_i}$$

        \section{Dropout}
            The Dropout Operator is used to prevent overfitting.  It does this by randomly selecting $\alpha$ percentage of values to set
            to zero. $\alpha$ is a hyperparameter that can be tuned, but is usually set between 0.5 and 0.2. In pure math, with $r$ representing
            a random value between 0 and 1,
            \[ \begin{cases} 
                0 & r < \alpha \\
                x\frac{1}{1 - \alpha} & r \ge \alpha \\
             \end{cases}
            \]
            When $r > \alpha$, we scale $x$ up by a factor of $x\frac{1}{1 - \alpha}$. In practice, this prevents neurons from finding
            relationships between neurons that are not significant, which tends to happen when networks are larger than they need to be or there
            is insufficient data. During test or inference, the dropout layer is turned off. 

            \paragraph{Gradient w.r.t. A}
            During backpropogation, gradients at locations which were set to 0 during the forward operation are also zeroed, and otherwise the gradients
            are scaled up by a factor of $g\frac{1}{1 - \alpha}$.

    \chapter{Activation Operators}

        \section{Sigmoid}
            The Sigmoid function is defined as $\sigma(x) = \frac{1}{1+e^{-x}}$ To find the gradient, we will apply the chain rule and simplify.
            \begin{align}
                \dfrac{d}{dx} \sigma(x) &= \dfrac{d}{dx} \left[ \dfrac{1}{1 + e^{-x}} \right]\\
                &= \dfrac{d}{dx} \left( 1 + \mathrm{e}^{-x} \right)^{-1} \\
                &= -(1 + e^{-x})^{-2}(-e^{-x}) \\
                &= \dfrac{e^{-x}}{\left(1 + e^{-x}\right)^2} \\
                &= \dfrac{1}{1 + e^{-x}\ } \cdot \dfrac{e^{-x}}{1 + e^{-x}}  \\
                &= \dfrac{1}{1 + e^{-x}\ } \cdot \dfrac{(1 + e^{-x}) - 1}{1 + e^{-x}}  \\
                &= \dfrac{1}{1 + e^{-x}\ } \cdot \left( \dfrac{1 + e^{-x}}{1 + e^{-x}} - \dfrac{1}{1 + e^{-x}} \right) \\
                &= \dfrac{1}{1 + e^{-x}\ } \cdot \left( 1 - \dfrac{1}{1 + e^{-x}} \right) \\
                &= \sigma(x) \cdot (1 - \sigma(x))
            \end{align}
            Therefore, we say that the gradient of the sigmoid function with respect to x is $\sigma(x) \cdot (1-\sigma(x))$.

        \section{Softmax}
            The Softmax operator "...converts a vector of length $K$ into a probability distribution of $K$ possible outcomes".\cite{wiki:Softmax_function} This operation
            is commonly used in classification, to produce a vector of probabilities. 
            $$\sigma(\overrightarrow{z})_i = \frac{e^{z_i}}{\sum_{j=0}^{K}e^{z_j}}$$
            In the example below, you can see that the softmax normalized the outputs to a range [0, 1] and the sum of these values is 1.  
            $$\sigma([1, 2, 3])_i = [0.09, 0.24, 0.67]$$
            In the following proof, I will refer to the input as $z$ and the output as $w$, both with a length of $K$. 

            \textbf{Proposition 1:}
            To define $\frac{\delta{w}}{\delta{z}}$ we will first define $\frac{\delta{w_i}}{\delta{z_j}}$, for which we will apply the quotient rule, defined below. 
            $$\frac{\delta}{\delta{x}}(\frac{u}{v}) = \frac{v\frac{\delta{u}}{\delta{x}} - u\frac{\delta{v}}{\delta{x}}}{v^2}$$
            Now we run into a problem. We want to substitute $z_j$ for $x$, but we will get different answers for $i\ne j$ and $i=j$. First, lets' define the $i=j$ case. 
            For simplicity, $\Sigma$ = $\sum_{j=0}^{K}e^{z_j}$.
            $$\frac{\delta{w_i}}{\delta{z_i}}(\sigma)_i = \frac{\Sigma \frac{\delta{e^{z_i}}}{\delta{z_i}} 
            - e^{z_i}\frac{\delta{\Sigma}}{\delta{z_i}}}{\Sigma^2} = \frac{e^{z_i}\Sigma - e^{z_i}e^{z_i}}{\Sigma^2}$$
            Simplify to an easier form. 
            $$= \frac{e^{z_i}}{\Sigma} \frac{\Sigma - e^{z_i}}{\Sigma} = \sigma_i(1 - \sigma_i)$$

            \textbf{Proposition 2:}
            In proposition 1 we defined $\frac{\delta{w_i}}{\delta{z_j}}$ when $i=j$, so now let us define the $i\ne j$ case. We will again apply the quotient rule.
            $$\frac{\delta{w_i}}{\delta{z_j}}(\sigma) = \frac{0- e^{z_i}\frac{\delta{\Sigma}}{\delta{z_i}}}{\Sigma^2} = \frac{-e^{z_j}e^{z_i}}{\Sigma^2} = 
            -\frac{e^{z_j}}{\Sigma} \frac{e^{z_i}}{\Sigma} = -\sigma_j \sigma_i$$

            Therefore, we can define $\frac{\delta{w_i}}{\delta{z_j}}$ as follows:
            \[ 
            \begin{cases} 
                i=j & \sigma_i(1 - \sigma_j) \\
                i\ne j & -\sigma_j \sigma_i
            \end{cases}
            \]


    \chapter{Higher Rank Operators}

        \section{Matmul}
            The Matmul operator computes the dot product of two matrices \textbf{A} and \textbf{B}. When \textbf{A} is an M x N 
            matrix, \textbf{B} is an N x P matrix, the product \textbf{C} is a M x P matrix defined as: 
            $$\sum_{i=0}^{M}\sum_{j=0}^{P}\boldsymbol{C}_{ij} = \sum_{k=0}^{N} \boldsymbol{A}_{ik}\boldsymbol{B}_{kj}$$
            Consider the following row-major matrices, where indices start at 0 and 0,0 is the top left. 
            $$
                A = \begin{bmatrix}1 & 2 & 3\\ 4 & 5 & 6 \end{bmatrix}
                B = \begin{bmatrix}1 & 2 \\ 3 & 4 \\ 5 & 6 \end{bmatrix}
                C = \begin{bmatrix} 22 & 28 \\ 49 & 62\end{bmatrix}
            $$
            \textbf{Proposition 1:}
            The dot product of A and B is denoted $C = A \cdot B$. The definition of an element $C_{ij}$ is the dot product of row vector $A_i$ and column vector $B_j$. 
            $$C_{ij} = \sum_{k=0}^{N} A_{ik}B_{kj}$$
            For example, to find $C_{00}$, this formula will expand to:
            $$C_{00} = A_{00} * B_{00} + A_{01} * B_{10} + A_{02} * B_{20}$$
            \textbf{Proposition 2:}
            To find the gradient of C w.r.t. an element of A or B, we will first differentiate the formula defined by proposition 1 by applying the product rule.
            $$\frac{\delta}{\delta{X}}(\sum_{k=0}^{N}A_{ik} B_{kj}) = \sum_{k=0}^{N} B_{kj} \frac{\delta{A_{ik}}}{\delta{X}} + A_{ik} \frac{\delta{B_{kj}}}{\delta{X}}$$ 
            Now substitute to find the gradient of $C_{ij}$ w.r.t. $A_{ik}$ and $B_{kj}$. We can remove the summation because we are finding the gradient for an element 
            of X instead of the entire vector X. 
            $$\frac{\delta}{\delta{A_{ik}}}(A_{ik} B_{kj}) = B_{kj} \frac{\delta{A_{ik}}}{\delta{A_{ik}}} + A_{ik} \frac{\delta{B_{kj}}}{\delta{A_{ik}}} = B_{kj}$$ 
            $$\frac{\delta}{\delta{B_{kj}}}(A_{ik} B_{kj}) = B_{kj} \frac{\delta{A_{ik}}}{\delta{B_{kj}}} + A_{ik} \frac{\delta{B_{kj}}}{\delta{B_{kj}}} = A_{ik}$$ 
            In plain english, the gradient of an index with respect to some $C_{ij}$ is the element in the corresponding vector it was multiplied by to get $C_{ij}$.  
            For example, the gradient
            of $A_{01}$ w.r.t. $C_{00}$ is $B_{10}$, since $A_{01}$ is multiplied by $B_{10}$ while calculating $C_{00}$. Also, $A_{01}$ was also multiplied by 
            $B_{11}$ while calculating
            $C_{01}$. Since we know all the elements $A_{01}$ was multiplied by, we can define the gradient of C w.r.t. $A_{01}$ to be $B_{10} + B_{11}$. 

            Take note of the fact that the gradient w.r.t. $A_{01}$ is the sum of the elements of row vector $B_1$. The same logic applies for e.g. the gradient 
            w.r.t. $B_{20}$, which is the sum of 
            the elements in column vector $A_2$. Therefore, we can define the gradient of C w.r.t. $A_{ik}$ as the sum of the columns of $B_k$ and the gradient 
            of C w.r.t. $A_{kj}$ as the sum of the rows of $A_k$. 
            $$\frac{\delta}{\delta{A_{ik}}} = \sum_{j=0}^{P} B_{kj}$$
            $$\frac{\delta}{\delta{B_{kj}}} = \sum_{i=0}^{M} A_{ik}$$
            \textbf{Proposition 3:}
            To calculate the gradient w.r.t. A we can take advantage of the axiom that A shares an axis N with B and axis M with C.  
            We can then take the dot product of some gradient matrix GC and the transpose of B. The same logic applies to the gradient w.r.t. B, 
            taking advantage of the axiom that B shares an axis N with A and axis P with C. Therefore, we can define the gradients as follows:
            $$GA = GC \cdot B^T$$ 
            $$GB = A^T \cdot GC$$
            Here are those operations written out: 
            $$GA = \begin{bmatrix} 1 & 1 \\ 1 & 1\end{bmatrix} \cdot \begin{bmatrix}1 & 3 & 5 \\ 2 & 4 & 6 \end{bmatrix} = \begin{bmatrix}3 & 7 & 11 \\ 3 & 7 & 11 \end{bmatrix}$$
            $$GB = \begin{bmatrix} 1 & 4 \\ 2 & 5 \\ 3 & 6 \end{bmatrix} \cdot \begin{bmatrix} 1 & 1 \\ 1 & 1 \end{bmatrix} = \begin{bmatrix}5 & 5 \\ 7 & 7 \\ 9 & 9 \end{bmatrix}$$

            This satisfies proposition 2, since GA is the sum of the rows of B, and GB is the sum of the rows in A. 

            GC here is the input gradient w.r.t. C. While backpropogating, these may be any value, but for this example case it is an identity matrix. GC is assumed to 
            have the same shape as C.  
            The output gradient GA has the same dimensions as A, and GB has the same dimensions as B. 

    \chapter{Loss Functions}

        \section{MSE}
            The Mean Squared Error function measures the average of the squares of the losses.  It is defined as follows, where $t$ and $p$ are 
            vectors of length $n$. $t$ is the target output and $p$ is the predicted output. 
            $$mse(t, p) = \frac{1}{n} \sum_{i=0}^{n} (t_i - p_i)^2$$ 
            
            \paragraph{Gradient}
                The MSE of an estimator is defined as below\cite{wiki:Mean_squared_error}.
                $$mse(\hat{\theta}) = E_\theta [(\hat{\theta} - \theta)^2]$$
                To find the gradient w.r.t. $\hat{\theta}$, we can apply the chain rule. 
                $$\frac{\delta}{\delta{\hat{\theta}}}((\hat{\theta} - \theta)^2) = 2(\hat{\theta} - \theta)$$
                Then we can rewrite this in terms of $t$ and $p$. 
                $$\frac{\delta}{\delta{\hat{p_i}}}((\hat{p_i} - t_i)^2) = 2(\hat{p_i} - t_i)$$
                Therefore, the gradient of the MSE of an estimator is defined as $2(\hat{p_i} - t_i)$.

        \section{MAE}
            The Mean Absolute Error function measures the average of the absolute value of the losses. It is defined as follows, where $t$ and $p$
            are vectors of length $n$. $t$ is the target output and $p$ is the predicted output. 
            $$mae(t, p) = \frac{1}{n} \sum_{i=0}^{n} |p_i - t_i|$$

            \paragraph{Gradient}
                The MAE of an estimator is defined below.
                $$mae(\hat{\theta}) = E_\theta[|\theta - \hat{\theta}|]$$

    \printbibliography

\end{document}